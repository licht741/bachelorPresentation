\documentclass[utf8,russian]{beamer}
\mode<presentation> {
  \usetheme{Madrid}
}

\usepackage{graphicx}
\usepackage{booktabs}

\usepackage[utf8]{inputenc}
\usepackage[T2A]{fontenc}
\usepackage[russian,english]{babel}

\setbeamertemplate{navigation symbols}{}

\setbeamercolor{footline}{}
\setbeamertemplate{footline}{
  \leavevmode%
  \hbox{
  \begin{beamercolorbox}[wd=.333333\paperwidth,ht=2.25ex,dp=1ex,center]{}%
    Н. М. Нежевский (Мехмат ЮФУ)
  \end{beamercolorbox}%
  \begin{beamercolorbox}[wd=.333333\paperwidth,ht=2.25ex,dp=1ex,center]{}%
    Ростов-на-Дону, 2016
  \end{beamercolorbox}%
  \begin{beamercolorbox}[wd=.333333\paperwidth,ht=2.25ex,dp=1ex,right]{}%
  Стр. \insertframenumber{} из \inserttotalframenumber \hspace*{2ex}
  \end{beamercolorbox}}%
  \vskip0pt%
}

\title{\small{Цветовая поддержка вывода консоли интерпретатора языка Haskell GHCi}}
\vspace{15pt}%
\author{\small{%
Н. М. Нежевский\\%
\emph{Направление подготовки:}~Фундаментальная информатика и \\информационные технологии\\%
\emph{Научный руководитель:} старший преподаватель В.~Н.~Брагилевский}\\%
\vspace{15pt}%
    Южный федеральный университет\\
    Институт математики, механики и компьютерных наук
    имени~И.\,И.\,Воровича%
}
\date{\small{Ростов-на-Дону, 2016}}

\begin{document}

\begin{frame}
\titlepage
\end{frame}

\begin{frame}
\frametitle{Содержание}
\tableofcontents
\end{frame}

\section{Постановка задачи}
\begin{frame}
\frametitle{Постановка задачи}
\begin{itemize}
  \item \textbf{Анализ возможных подходов к реализации подсветки вывода.}
  \item \textbf{Изучение программной структуры GHCi.}
  \item \textbf{Программная реализация цветовой поддержки вывода интерпретатора GHCi.} 
\end{itemize}
\end{frame}

\section{Программная реализация}
\begin{frame}
\frametitle{Программная реализация}

\begin{block}{Реализация на основе внешней обработки}
Получена программная реализация, обрабатывающая консольный вывод GHCi
\end{block}

\begin{block}{Алгоритм работы}
\begin{enumerate}
  \item Внешняя программа-обработчик принимает пользовательский ввод
  \item Этот пользовательский ввод передаётся в GHCi
  \item Программа-обработчик перехватывает результат вывода GHCi
  \item Строковый вывод окрашивается добавлением ANSI символов и возвращается пользователю
\end{enumerate}
\end{block}
\end{frame}
\onslide<1>
\begin{frame}{Анализ полученной реализации}
\begin{block}{Недостатки реализации}
\begin{enumerate}
  \item Окрашивание возможно только для стандартных типов и классов типов
  \item Не поддерживается окрашивание переменных
\end{enumerate}
\end{block}

\onslide<2>
\begin{block}{Вывод}
Полученная реализация решает поставленную задачу, однако имеет ряд ограничений
\end{block}

\end{frame}

\end{document}